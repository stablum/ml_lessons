\documentclass{beamer}
\usepackage[utf8]{inputenc}
\usepackage{amsmath,amssymb}

\usetheme{Madrid}        % You can change the theme
\usecolortheme{default}  % You can change the color theme

\title{Mean Squared Error (MSE) Tutorial}
\author{}
\date{\today}

\begin{document}

%%%%%%%%%%%%%%%%%%%%%%%%%%%%%%%%%%%%%%%%%%%%%%%%%%%%%%%%%%%%%%%
% Title Slide
%%%%%%%%%%%%%%%%%%%%%%%%%%%%%%%%%%%%%%%%%%%%%%%%%%%%%%%%%%%%%%%
\begin{frame}
  \titlepage
\end{frame}

%%%%%%%%%%%%%%%%%%%%%%%%%%%%%%%%%%%%%%%%%%%%%%%%%%%%%%%%%%%%%%%
% Frame: MSE Formula
%%%%%%%%%%%%%%%%%%%%%%%%%%%%%%%%%%%%%%%%%%%%%%%%%%%%%%%%%%%%%%%
\begin{frame}{Formula}
\[
J_{\text{MSE}} = \frac{1}{n} \sum_{i=1}^{n} \bigl(y_i - \hat{y}_i\bigr)^2
\]

\begin{itemize}
  \item \(y_i\) are the true target values
  \item \(\hat{y}_i\) are the predicted values
  \item \(n\) is the number of observations
  \item \(J_{\text{MSE}}\) denotes the MSE-based loss function (here we use \(J\) to represent any loss/cost function)
\end{itemize}
\end{frame}

%%%%%%%%%%%%%%%%%%%%%%%%%%%%%%%%%%%%%%%%%%%%%%%%%%%%%%%%%%%%%%%
% Frame: Best for
%%%%%%%%%%%%%%%%%%%%%%%%%%%%%%%%%%%%%%%%%%%%%%%%%%%%%%%%%%%%%%%
\begin{frame}{Best for}
\begin{itemize}
  \item \textbf{Penalizing large errors heavily:} Squaring the differences means that larger errors have a disproportionately higher impact on the overall cost.
\end{itemize}
\end{frame}

%%%%%%%%%%%%%%%%%%%%%%%%%%%%%%%%%%%%%%%%%%%%%%%%%%%%%%%%%%%%%%%
% Frame: Characteristics
%%%%%%%%%%%%%%%%%%%%%%%%%%%%%%%%%%%%%%%%%%%%%%%%%%%%%%%%%%%%%%%
\begin{frame}{Characteristics}
\begin{itemize}
  \item \textbf{Differentiable and easy to compute:} The MSE-based cost function is smooth and lends itself well to gradient-based optimization.
  \item \textbf{Sensitive to outliers:} Squaring amplifies large errors, making the metric more sensitive to outliers.
  \item \textbf{Encourages predictions close to the mean:} Especially when the target distribution is unimodal, the MSE criterion drives predictions toward the mean of the targets.
\end{itemize}
\end{frame}

%%%%%%%%%%%%%%%%%%%%%%%%%%%%%%%%%%%%%%%%%%%%%%%%%%%%%%%%%%%%%%%
% Frame: Interpretations - Gaussian Distribution
%%%%%%%%%%%%%%%%%%%%%%%%%%%%%%%%%%%%%%%%%%%%%%%%%%%%%%%%%%%%%%%
\begin{frame}{Gaussian Distribution}
\textbf{PDF of a Gaussian (Normal) distribution with mean \(\mu\) and variance \(\sigma^2\) for a variable \(x\):}
\[
\mathcal{L}(x; \mu, \sigma^2) = \frac{1}{\sqrt{2\pi\sigma^2}} 
\exp\!\Biggl( -\frac{(x - \mu)^2}{2\sigma^2} \Biggr),
\]
where we use \(\mathcal{L}\) to indicate the likelihood function.
\end{frame}

%%%%%%%%%%%%%%%%%%%%%%%%%%%%%%%%%%%%%%%%%%%%%%%%%%%%%%%%%%%%%%%
% Frame: Probabilistic Perspective (Derivation)
%%%%%%%%%%%%%%%%%%%%%%%%%%%%%%%%%%%%%%%%%%%%%%%%%%%%%%%%%%%%%%%
\begin{frame}{Probabilistic Perspective (Derivation)}
Assuming the prediction errors follow a Gaussian distribution with mean 0 and variance \(\sigma^2\), 
the likelihood \(\mathcal{L}\) for a single observation \(y_i\) given the predicted value \(\hat{y}_i\) is:

\[
\mathcal{L}(y_i \mid \hat{y}_i) = 
\frac{1}{\sqrt{2\pi\sigma^2}} 
\exp\Bigl( -\frac{(y_i - \hat{y}_i)^2}{2\sigma^2} \Bigr).
\]

\textbf{Steps:}
\begin{enumerate}
\item \(\displaystyle 
\log \mathcal{L}(y_i \mid \hat{y}_i) = 
\log\Bigl(\frac{1}{\sqrt{2\pi\sigma^2}}\Bigr) + 
\log\Bigl(\exp\Bigl( -\tfrac{(y_i - \hat{y}_i)^2}{2\sigma^2}\Bigr)\Bigr)\)

\item \(\displaystyle 
\log \mathcal{L}(y_i \mid \hat{y}_i) = 
-\frac{1}{2}\log(2\pi\sigma^2) 
-\frac{(y_i - \hat{y}_i)^2}{2\sigma^2}\)

\item \(\displaystyle 
J_i = -\log \mathcal{L}(y_i \mid \hat{y}_i) = 
\frac{(y_i - \hat{y}_i)^2}{2\sigma^2} + 
\frac{1}{2}\log(2\pi\sigma^2)\)
\end{enumerate}
\end{frame}

%%%%%%%%%%%%%%%%%%%%%%%%%%%%%%%%%%%%%%%%%%%%%%%%%%%%%%%%%%%%%%%
% Frame: Single-Point vs. Full Likelihood
%%%%%%%%%%%%%%%%%%%%%%%%%%%%%%%%%%%%%%%%%%%%%%%%%%%%%%%%%%%%%%%
\begin{frame}{Single-Point vs. Full Likelihood}
\begin{itemize}
  \item Usually, \textbf{likelihood} refers to the joint function of \emph{all} data points.
  \item However, each individual term in the product (or sum in log space) is often called the likelihood contribution of a single observation.
  \item We can thus show the negative log-likelihood derivation \emph{per data point}, then combine them for the full dataset.
\end{itemize}
\end{frame}

%%%%%%%%%%%%%%%%%%%%%%%%%%%%%%%%%%%%%%%%%%%%%%%%%%%%%%%%%%%%%%%
% Frame: Negative Log-Likelihood for the Entire Dataset
%%%%%%%%%%%%%%%%%%%%%%%%%%%%%%%%%%%%%%%%%%%%%%%%%%%%%%%%%%%%%%%
\begin{frame}{Negative Log-Likelihood for the Entire Dataset}
For \(n\) independent observations, the joint likelihood is the product of individual likelihoods:
\[
J_{\text{NLL}} = -\log \mathcal{L}\bigl(\{y_i\}\mid\{\hat{y}_i\}\bigr) 
= \sum_{i=1}^{n} \Bigl[\frac{(y_i - \hat{y}_i)^2}{2\sigma^2} 
+ \frac{1}{2} \log (2 \pi \sigma^2)\Bigr].
\]

Ignoring the constant term \(\tfrac{1}{2} n \log (2\pi\sigma^2)\) gives:
\[
J_{\text{NLL}} = \frac{1}{2\sigma^2} \sum_{i=1}^n (y_i - \hat{y}_i)^2 
\quad + \text{(const.)}.
\]
\textbf{Minimizing this w.r.t. \(\hat{y}_i\) \(\Longrightarrow\) Minimizing MSE.}
\end{frame}

%%%%%%%%%%%%%%%%%%%%%%%%%%%%%%%%%%%%%%%%%%%%%%%%%%%%%%%%%%%%%%%
% Frame: Geometric Perspective (Derivation)
%%%%%%%%%%%%%%%%%%%%%%%%%%%%%%%%%%%%%%%%%%%%%%%%%%%%%%%%%%%%%%%
\begin{frame}{Geometric Perspective (Derivation)}
\[
\| y - \hat{y} \|_2^2 = \sum_{i=1}^{n} (y_i - \hat{y}_i)^2.
\]

\begin{itemize}
  \item The MSE corresponds to squared Euclidean (L2) distance.
  \item Minimizing \(\| y - \hat{y} \|_2^2\) means finding the point \(\hat{y}\) closest to the actual target \(y\).
\end{itemize}
\end{frame}

%%%%%%%%%%%%%%%%%%%%%%%%%%%%%%%%%%%%%%%%%%%%%%%%%%%%%%%%%%%%%%%
% Frame: Connection to Linear Regression
%%%%%%%%%%%%%%%%%%%%%%%%%%%%%%%%%%%%%%%%%%%%%%%%%%%%%%%%%%%%%%%
\begin{frame}{Connection to Linear Regression}
\begin{itemize}
  \item MSE is the standard cost function in \textbf{Ordinary Least Squares (OLS)} regression.
  \item Linear model: \( y = X\beta + \varepsilon \) 
    where \(X\) is the design matrix, \(\beta\) is the parameter vector, and \(\varepsilon\) is noise.
  \item Minimizing 
    \(\displaystyle J_{\text{MSE}}(\beta) = \sum_{i=1}^{n} (y_i - X_i \beta)^2\)
    leads to the OLS estimate.
  \item Closed-form solution (assuming invertibility): 
    \(\displaystyle \hat{\beta} = (X^T X)^{-1} X^T y.\)
\end{itemize}
\end{frame}

%%%%%%%%%%%%%%%%%%%%%%%%%%%%%%%%%%%%%%%%%%%%%%%%%%%%%%%%%%%%%%%
% Frame: Detailed Derivation of OLS Solution (Matrix)
%%%%%%%%%%%%%%%%%%%%%%%%%%%%%%%%%%%%%%%%%%%%%%%%%%%%%%%%%%%%%%%
\begin{frame}{Detailed Derivation of the OLS Solution}
\textbf{1. Cost function:}
\[
J_{\text{MSE}}(\beta) = \sum_{i=1}^n \bigl(y_i - X_i \beta\bigr)^2,
\]
in matrix form:
\[
J_{\text{MSE}}(\beta) = (y - X\beta)^T (y - X\beta).
\]

\textbf{2. Take the gradient w.r.t. \(\beta\):}
\[
(y - X\beta)^T (y - X\beta) 
= y^T y - 2\,y^T X\beta + \beta^T X^T X\beta.
\]
\[
\nabla_{\beta} J_{\text{MSE}}(\beta) 
= -2 X^T y + 2 X^T X\beta.
\]

\textbf{3. Set the gradient to zero (Normal Equations):}
\[
X^T X \hat{\beta} = X^T y.
\]

\textbf{4. Solve for \(\hat{\beta}\):}
\[
\hat{\beta} = (X^T X)^{-1} X^T y.
\]
\end{frame}

%%%%%%%%%%%%%%%%%%%%%%%%%%%%%%%%%%%%%%%%%%%%%%%%%%%%%%%%%%%%%%%
% Frame: Summation-Based Derivation (Without Matrix Algebra)
%%%%%%%%%%%%%%%%%%%%%%%%%%%%%%%%%%%%%%%%%%%%%%%%%%%%%%%%%%%%%%%
\begin{frame}{Summation-Based Derivation (Without Matrix Algebra)}
\textbf{Model:}
\[
\hat{y}_i = \beta_0 + \sum_{j=1}^d \beta_j \, x_{ij}.
\]

\textbf{MSE cost:}
\[
J_{\text{MSE}}(\beta_0, \beta_1, \dots, \beta_d) 
= \sum_{i=1}^n \bigl(y_i - \hat{y}_i\bigr)^2
= \sum_{i=1}^n 
\Bigl(y_i - \beta_0 - \sum_{j=1}^d \beta_j x_{ij}\Bigr)^2.
\]

\textbf{Partial derivatives:}
\[
\frac{\partial J_{\text{MSE}}}{\partial \beta_0} = 
\sum_{i=1}^n -2\Bigl(y_i - \beta_0 
- \sum_{j=1}^d \beta_j x_{ij}\Bigr) = 0,
\]
\[
\frac{\partial J_{\text{MSE}}}{\partial \beta_k} = 
\sum_{i=1}^n -2\Bigl(y_i - \beta_0 
- \sum_{j=1}^d \beta_j x_{ij}\Bigr)\, x_{ik} = 0\quad (k \ge 1).
\]

\textbf{Solve the resulting \((d+1)\)-equation system:} 
equivalent to the Normal Equation solution.
\end{frame}

%%%%%%%%%%%%%%%%%%%%%%%%%%%%%%%%%%%%%%%%%%%%%%%%%%%%%%%%%%%%%%%
% Frame: Gradient-Based Optimization
%%%%%%%%%%%%%%%%%%%%%%%%%%%%%%%%%%%%%%%%%%%%%%%%%%%%%%%%%%%%%%%
\begin{frame}{Gradient-Based Optimization}
\[
\frac{\partial J_{\text{MSE}}}{\partial \hat{y}_i} 
= \frac{2}{n} (\hat{y}_i - y_i).
\]

\begin{itemize}
  \item Because \(J_{\text{MSE}}\) is smooth and differentiable, 
        it's well-suited for gradient-based methods.
  \item In parametric models (e.g., neural networks), 
        apply the chain rule to backprop through parameters.
\end{itemize}
\end{frame}

%%%%%%%%%%%%%%%%%%%%%%%%%%%%%%%%%%%%%%%%%%%%%%%%%%%%%%%%%%%%%%%
% Frame: Relationship to R^2
%%%%%%%%%%%%%%%%%%%%%%%%%%%%%%%%%%%%%%%%%%%%%%%%%%%%%%%%%%%%%%%
\begin{frame}{Relationship to \(\boldsymbol{R^2}\)}
\[
R^2 = 1 - \frac{\sum_{i=1}^{n} (y_i - \hat{y}_i)^2}
{\sum_{i=1}^{n} (y_i - \bar{y})^2}.
\]

\begin{itemize}
  \item Reducing \(\sum_{i=1}^{n} (y_i - \hat{y}_i)^2\) increases \(R^2\).
  \item A higher \(R^2\) indicates a better fit of the model to the data.
\end{itemize}
\end{frame}

\end{document}

