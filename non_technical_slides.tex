\documentclass{beamer}
\usepackage[utf8]{inputenc}
\usepackage{graphicx}

\usetheme{Madrid}
\usecolortheme{default}

\title{Understanding Mean Squared Error: A Non-Technical Approach}
\author{}
\date{\today}

\begin{document}

%%%%%%%%%%%%%%%%%%%%%%%%%%%%%%%%%%%%%%%%%%%%%%%%%%%%%%%%%%%%%%%
% Title Slide
%%%%%%%%%%%%%%%%%%%%%%%%%%%%%%%%%%%%%%%%%%%%%%%%%%%%%%%%%%%%%%%
\begin{frame}
  \titlepage
\end{frame}

%%%%%%%%%%%%%%%%%%%%%%%%%%%%%%%%%%%%%%%%%%%%%%%%%%%%%%%%%%%%%%%
% Frame: Introduction
%%%%%%%%%%%%%%%%%%%%%%%%%%%%%%%%%%%%%%%%%%%%%%%%%%%%%%%%%%%%%%%
\begin{frame}{What Is Mean Squared Error?}
  \begin{itemize}
    \item Mean Squared Error (MSE) is a way to measure how well predictions match reality
    \item Think of it as a "wrongness score" - the higher the score, the worse the predictions
    \item It's one of the most common ways to evaluate prediction models in data science
    \item Today, we'll understand what it means and why it matters - without complex math!
  \end{itemize}
\end{frame}

%%%%%%%%%%%%%%%%%%%%%%%%%%%%%%%%%%%%%%%%%%%%%%%%%%%%%%%%%%%%%%%
% Frame: Real-World Analogy
%%%%%%%%%%%%%%%%%%%%%%%%%%%%%%%%%%%%%%%%%%%%%%%%%%%%%%%%%%%%%%%
\begin{frame}{A Real-World Analogy}
  Imagine you're an archer shooting at a target:
  \begin{itemize}
    \item Each arrow represents a prediction
    \item The bullseye represents the true value you're trying to predict
    \item MSE measures how far your arrows land from the bullseye
    \item The farther away your arrows land, the higher your error score
    \item But MSE doesn't just measure distance - it emphasizes big misses
  \end{itemize}
\end{frame}

%%%%%%%%%%%%%%%%%%%%%%%%%%%%%%%%%%%%%%%%%%%%%%%%%%%%%%%%%%%%%%%
% Frame: How It Works - Simply
%%%%%%%%%%%%%%%%%%%%%%%%%%%%%%%%%%%%%%%%%%%%%%%%%%%%%%%%%%%%%%%
\begin{frame}{How MSE Works - Simply Explained}
  \begin{enumerate}
    \item Calculate how far each prediction is from the actual value
    \item Square each of these differences (multiply it by itself)
    \item Find the average of all these squared differences
  \end{enumerate}
  
  \vspace{0.5cm}
  That's it! The final number is your MSE score.
\end{frame}

%%%%%%%%%%%%%%%%%%%%%%%%%%%%%%%%%%%%%%%%%%%%%%%%%%%%%%%%%%%%%%%
% Frame: Why Square the Errors?
%%%%%%%%%%%%%%%%%%%%%%%%%%%%%%%%%%%%%%%%%%%%%%%%%%%%%%%%%%%%%%%
\begin{frame}{Why Square the Errors?}
  When we square the differences between predictions and actual values:
  
  \begin{itemize}
    \item All errors become positive (no negatives canceling out positives)
    \item Larger errors are penalized much more heavily
  \end{itemize}
  
  \vspace{0.5cm}
  Example: Which is worse?
  \begin{itemize}
    \item Making 10 predictions that are each off by 1 unit
    \item Making 9 perfect predictions but 1 prediction that's off by 10 units
  \end{itemize}
  
  With MSE, the second case is considered worse! A single big mistake counts more than many small ones.
\end{frame}

%%%%%%%%%%%%%%%%%%%%%%%%%%%%%%%%%%%%%%%%%%%%%%%%%%%%%%%%%%%%%%%
% Frame: Example - Weather Forecasting
%%%%%%%%%%%%%%%%%%%%%%%%%%%%%%%%%%%%%%%%%%%%%%%%%%%%%%%%%%%%%%%
\begin{frame}{Example: Weather Forecasting}
  Imagine predicting temperatures for a week:
  
  \begin{tabular}{lcc}
    \hline
    Day & Actual Temp (°F) & Predicted Temp (°F) \\
    \hline
    Monday & 75 & 73 \\
    Tuesday & 78 & 80 \\
    Wednesday & 82 & 79 \\
    Thursday & 79 & 77 \\
    Friday & 77 & 82 \\
    \hline
  \end{tabular}
  
  \vspace{0.5cm}
  The MSE would tell us how accurate our weather model is, with higher penalties for days when we were way off.
\end{frame}

%%%%%%%%%%%%%%%%%%%%%%%%%%%%%%%%%%%%%%%%%%%%%%%%%%%%%%%%%%%%%%%
% Frame: When MSE Shines
%%%%%%%%%%%%%%%%%%%%%%%%%%%%%%%%%%%%%%%%%%%%%%%%%%%%%%%%%%%%%%%
\begin{frame}{When MSE Shines}
  MSE works especially well when:
  
  \begin{itemize}
    \item Big errors are particularly problematic
    \item Your data doesn't have many extreme outliers
    \item The consequences of errors grow with their size
  \end{itemize}
  
  \vspace{0.5cm}
  
  Real-world examples:
  \begin{itemize}
    \item Predicting house prices (a $50K error is much worse than five $10K errors)
    \item Estimating delivery times (being 5 hours late is worse than being 1 hour late on 5 deliveries)
    \item Financial forecasting (where large prediction errors could be very costly)
  \end{itemize}
\end{frame}

%%%%%%%%%%%%%%%%%%%%%%%%%%%%%%%%%%%%%%%%%%%%%%%%%%%%%%%%%%%%%%%
% Frame: The Outlier Problem
%%%%%%%%%%%%%%%%%%%%%%%%%%%%%%%%%%%%%%%%%%%%%%%%%%%%%%%%%%%%%%%
\begin{frame}{The Outlier Problem}
  MSE's sensitivity to large errors has a downside:
  
  \begin{itemize}
    \item A single extreme outlier can dominate the entire error score
    \item This might make an otherwise good model look terrible
    \item Example: A house price prediction model performs well on 99% of houses but is way off on one mansion
  \end{itemize}
  
  \vspace{0.5cm}
  
  This is why data scientists sometimes use alternatives like Mean Absolute Error when outliers are a concern.
\end{frame}

%%%%%%%%%%%%%%%%%%%%%%%%%%%%%%%%%%%%%%%%%%%%%%%%%%%%%%%%%%%%%%%
% Frame: Driving Toward the Middle
%%%%%%%%%%%%%%%%%%%%%%%%%%%%%%%%%%%%%%%%%%%%%%%%%%%%%%%%%%%%%%%
\begin{frame}{Driving Toward the Middle}
  An interesting property of MSE:
  
  \begin{itemize}
    \item If you had to make the same prediction for everything, what would you choose?
    \item With MSE, the best single prediction is the average of all actual values
  \end{itemize}
  
  \vspace{0.5cm}
  
  Example: If you had to guess everyone's age in a room with one number:
  \begin{itemize}
    \item Using MSE, the average age would be your best bet
    \item This explains why basic prediction models often start with averages
  \end{itemize}
\end{frame}

%%%%%%%%%%%%%%%%%%%%%%%%%%%%%%%%%%%%%%%%%%%%%%%%%%%%%%%%%%%%%%%
% Frame: Connection to Linear Regression
%%%%%%%%%%%%%%%%%%%%%%%%%%%%%%%%%%%%%%%%%%%%%%%%%%%%%%%%%%%%%%%
\begin{frame}{Connection to Linear Regression}
  Linear regression (finding the "best fit line") is directly tied to MSE:
  
  \begin{itemize}
    \item When we say "best fit line," we mean the line that minimizes the MSE
    \item The regression line is positioned to make the squared vertical distances as small as possible
    \item This is why it's called "least squares" regression
  \end{itemize}
  
  \vspace{0.5cm}
  
  In essence, the familiar best-fit line you see through scattered points is simply the line that minimizes the MSE!
\end{frame}

%%%%%%%%%%%%%%%%%%%%%%%%%%%%%%%%%%%%%%%%%%%%%%%%%%%%%%%%%%%%%%%
% Frame: The Normal Distribution Connection
%%%%%%%%%%%%%%%%%%%%%%%%%%%%%%%%%%%%%%%%%%%%%%%%%%%%%%%%%%%%%%%
\begin{frame}{Behind the Scenes: The Bell Curve Connection}
  There's a deeper reason why MSE is so common:
  
  \begin{itemize}
    \item It assumes errors follow a "bell curve" (normal distribution)
    \item Many natural phenomena do follow bell curves
    \item Examples: Heights of people, measurement errors, natural variations in many biological processes
  \end{itemize}
  
  \vspace{0.5cm}
  
  When data naturally varies according to a bell curve, MSE is theoretically the most efficient way to measure error!
\end{frame}

%%%%%%%%%%%%%%%%%%%%%%%%%%%%%%%%%%%%%%%%%%%%%%%%%%%%%%%%%%%%%%%
% Frame: How Good Is My Model?
%%%%%%%%%%%%%%%%%%%%%%%%%%%%%%%%%%%%%%%%%%%%%%%%%%%%%%%%%%%%%%%
\begin{frame}{How Good Is My Model?}
  MSE helps answer this question, but with a twist:
  
  \begin{itemize}
    \item A related measure called R-squared tells what percentage of variation your model explains
    \item R-squared of 0.75 means your model explains 75\% of the variation in the data
    \item The closer to 1 (or 100\%), the better your model fits the data
  \end{itemize}
  
  \vspace{0.5cm}
  
  This gives a more intuitive sense of model quality than raw MSE numbers.
\end{frame}

%%%%%%%%%%%%%%%%%%%%%%%%%%%%%%%%%%%%%%%%%%%%%%%%%%%%%%%%%%%%%%%
% Frame: Practical Applications
%%%%%%%%%%%%%%%%%%%%%%%%%%%%%%%%%%%%%%%%%%%%%%%%%%%%%%%%%%%%%%%
\begin{frame}{Practical Applications}
  MSE is used in countless real-world scenarios:
  
  \begin{itemize}
    \item Stock price prediction (financial services)
    \item Energy consumption forecasting (utilities)
    \item Inventory management (retail)
    \item Patient outcome prediction (healthcare)
    \item Traffic flow prediction (transportation)
    \item Image and video compression (technology)
  \end{itemize}
  
  Essentially, whenever we need a model to make numerical predictions, MSE is likely in the picture!
\end{frame}

%%%%%%%%%%%%%%%%%%%%%%%%%%%%%%%%%%%%%%%%%%%%%%%%%%%%%%%%%%%%%%%
% Frame: When NOT to Use MSE
%%%%%%%%%%%%%%%%%%%%%%%%%%%%%%%%%%%%%%%%%%%%%%%%%%%%%%%%%%%%%%%
\begin{frame}{When NOT to Use MSE}
  MSE isn't always the right choice:
  
  \begin{itemize}
    \item When you're predicting categories (like "spam" vs "not spam"), not numbers
    \item When all errors should be treated equally (a small miss is as bad as a large one)
    \item When your data has many extreme outliers
    \item When the units of your prediction are important to interpret (MSE units are squared!)
  \end{itemize}
  
  In these cases, other error measures may be more appropriate.
\end{frame}

%%%%%%%%%%%%%%%%%%%%%%%%%%%%%%%%%%%%%%%%%%%%%%%%%%%%%%%%%%%%%%%
% Frame: Takeaways
%%%%%%%%%%%%%%%%%%%%%%%%%%%%%%%%%%%%%%%%%%%%%%%%%%%%%%%%%%%%%%%
\begin{frame}{Key Takeaways}
  Remember these main points about MSE:
  
  \begin{itemize}
    \item It measures prediction error by averaging squared differences
    \item It penalizes large errors much more than small ones
    \item It's the foundation of many common statistical methods
    \item It works best when errors naturally follow a bell curve
    \item It's just one of many possible error measures - choose the right tool for your needs
  \end{itemize}
\end{frame}

%%%%%%%%%%%%%%%%%%%%%%%%%%%%%%%%%%%%%%%%%%%%%%%%%%%%%%%%%%%%%%%
% Frame: Questions
%%%%%%%%%%%%%%%%%%%%%%%%%%%%%%%%%%%%%%%%%%%%%%%%%%%%%%%%%%%%%%%
\begin{frame}{Questions?}
  \centering
  \huge Thank you for your attention!
  
  \vspace{1cm}
  
  \large Any questions?
\end{frame}

\end{document}